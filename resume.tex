%%%%%%%%%%%%%%%%%%%%%%%%%%%%%%%%%%%%%%%
% Deedy - One Page Two Column Resume
% LaTeX Template
% Version 1.2 (16/9/2014)
%
% Original author:
% Debarghya Das (http://debarghyadas.com)
%
% Original repository:
% https://github.com/deedydas/Deedy-Resume
%
% IMPORTANT: THIS TEMPLATE NEEDS TO BE COMPILED WITH XeLaTeX
%
% This template uses several fonts not included with Windows/Linux by
% default. If you get compilation errors saying a font is missing, find the line
% on which the font is used and either change it to a font included with your
% operating system or comment the line out to use the default font.
%
%%%%%%%%%%%%%%%%%%%%%%%%%%%%%%%%%%%%%%
%
% TODO:
% 1. Integrate biber/bibtex for article citation under publications.
% 2. Figure out a smoother way for the document to flow onto the next page.
% 3. Add styling information for a "Projects/Hacks" section.
% 4. Add location/address information
% 5. Merge OpenFont and MacFonts as a single sty with options.
%
%%%%%%%%%%%%%%%%%%%%%%%%%%%%%%%%%%%%%%
%
% CHANGELOG:
% v1.1:
% 1. Fixed several compilation bugs with \renewcommand
% 2. Got Open-source fonts (Windows/Linux support)
% 3. Added Last Updated
% 4. Move Title styling into .sty
% 5. Commented .sty file.
%
%%%%%%%%%%%%%%%%%%%%%%%%%%%%%%%%%%%%%%%
%
% Known Issues:
% 1. Overflows onto second page if any column's contents are more than the
% vertical limit
% 2. Hacky space on the first bullet point on the second column.
%
%%%%%%%%%%%%%%%%%%%%%%%%%%%%%%%%%%%%%%


\documentclass[]{deedy-resume-openfont}
\newcommand\tab[1][0.5cm]{\hspace*{#1}}
\usepackage{fancyhdr}
\usepackage{hyperref}

\pagestyle{fancy}
\fancyhf{}

\begin{document}

%%%%%%%%%%%%%%%%%%%%%%%%%%%%%%%%%%%%%%
%
%     LAST UPDATED DATE
%
%%%%%%%%%%%%%%%%%%%%%%%%%%%%%%%%%%%%%%
% \lastupdated % hide last updated date

%%%%%%%%%%%%%%%%%%%%%%%%%%%%%%%%%%%%%%
%
%     TITLE NAME
%
%%%%%%%%%%%%%%%%%%%%%%%%%%%%%%%%%%%%%%


\begin{minipage}{0.6\textwidth}
    \namesection{Mao-Lin}{Wang}{}
\end{minipage}
\hfill
\begin{minipage}{0.4\textwidth}
    \begin{tightemize}
        \item[] \urlstyle{same}\href{https://focaaby.com/}{https://focaaby.com/}
        \item[] +886-926-633005 | \href{mailto:focaaby@gmail.com}{focaaby@gmail.com}
    \end{tightemize}
\end{minipage}

\par\noindent\rule{\textwidth}{0.5pt} % horizontal line

%%%%%%%%%%%%%%%%%%%%%%%%%%%%%%%%%%%%%%
%
%     COLUMN ONE
%
%%%%%%%%%%%%%%%%%%%%%%%%%%%%%%%%%%%%%%

\begin{minipage}[t]{1\textwidth}


%%%%%%%%%%%%%%%%%%%%%%%%%%%%%%%%%%%%%%
%     SUMMARY
%%%%%%%%%%%%%%%%%%%%%%%%%%%%%%%%%%%%%%

\section{Summary}
\begin{summary}
    Experienced DevOps engineer with experiences for over ten years in Linux and over eight years of expertise in Kubernetes, CI/CD, and leadership. Demonstrated ability to solve complex infrastructure problems independently.
\end{summary}

\section{Skill}
\vspace{\topsep}
\begin{tightemize}
    \item \textbf{\emph{Linux}} 10+ years | System admin and architecture including building, managing, maintaining, monitoring, and tuning, SRE.
    \item \textbf{\emph{Education}} 10+ years | Training, study circle host, workshops, companies, In University as a teaching assistant since 2013.
    \item \textbf{\emph{DevOps}} 8+ years | Distributed systems, Container, Docker, Kubernetes, GitHub/GitLab, Ansible, Terraform, Prometheus, AWS Infra.
    \item \textbf{\emph{Software Engineering}}  8+ years | JavaScript (Vanilla JS, Node.js).
\end{tightemize}

%%%%%%%%%%%%%%%%%%%%%%%%%%%%%%%%%%%%%%
%     EXPERIENCE
%%%%%%%%%%%%%%%%%%%%%%%%%%%%%%%%%%%%%%

\section{Experience}


\runsubsection{SHOPLINE}
\descript{Senior Cloud Engineer}
\null\hfill\location{Aug 2024 – present | Taipei, Taiwan}
\begin{tightemize}
    \item Optimized OpenResty stacks with zero downtime, reducing operational costs.
    \item Upgraded Amazon EKS clusters with zero downtime to enhance system reliability.
\end{tightemize}
\sectionsep


\runsubsection{National Institute of Cyber Security}
\descript{Researcher}
\null\hfill\location{Oct 2023 - Jul 2024 | Taipei, Taiwan}
\begin{tightemize}
    \item Advised government agencies on digital resilience and high availability architectures, ensuring robust and continuous operations.
    \item Developed and implemented comprehensive cloud security guidelines for government agencies.
    \item Created and standardized CI/CD guidelines for GitHub/GitLab to streamline software development and deployment processes within government agencies.
    \item Promoted and provided guidelines for Software Bill of Materials (SBOM) to enhance software supply chain transparency and security in government agencies.
\end{tightemize}
\sectionsep

\runsubsection{AppWorks School}
\descript{SRE Adjunct Lecturer}
\null\hfill\location{Sep 2023 – May 2024 | Taipei, Taiwan}
\begin{tightemize}
    \item Designed and delivered a Kubernetes course for students, covering topics such as Kubernetes architecture, deployment, and troubleshooting workshops.
\end{tightemize}
\sectionsep


\runsubsection{Amazon Web Services}
\descript{Cloud Support Engineer II/I/Associate}
\null\hfill\location{Jul 2019 - Aug 2023 | Ireland / Taiwan}
\begin{tightemize}
    \item Mandarin Engineer of the Year(2022, EMEA team).
    \item Certification of AWS EKS Subject Matter Expert.
    \item Creator of EKS operations helper(Tampermonkey script) to streamline workflow.
    \item Mentor of DevOps and Security domain services.
    \item Interviewer of DevOps domain.
    \item Received over 300+ 5-star ratings from customer feedback.
    \item Maintainer of Accessibility and Usability of EMEA language and Mandarin team pages in internal Knowledge Management system.
    \item Troubleshooting and resolving business critical issues for customers in various industries about Kubernetes and CI/CD domain services.
\end{tightemize}
\sectionsep

\runsubsection{Bridgewell}
\descript{DevOps \& Project Manager}
\null\hfill\location{Sep 2018 – June 2019 | Taipei, Taiwan}
\begin{tightemize}
    \item Spearheaded the building of Kubernetes from scratch.
    \item Streamlined the CI/CD pipeline through GitLab templates optimization.
    \item Optimized and reduced overall product deliverable package size by 80\% through Docker image optimization.
    \item Led the implementation of an internal financial system as project manager and frontend engineer, collaborating with multiple departments to ensure successful delivery of project requirements.
\end{tightemize}
\sectionsep

\runsubsection{Taiwan Stock Exchange}
\descript{Research Assistant}
\null\hfill\location{May 2017 – Dec 2017 | Taipei, Taiwan}
\begin{tightemize}
    \item Revamped a high-availability infrastructure with Docker containers to mitigate the single point of failure (SPOF) issue with Kafka and Zookeeper, ensuring efficient high-throughput message queue operations.
    \item Automated the CI/CD pipeline for testing by implementing an Ansible playbook with different environment variables.
\end{tightemize}
\sectionsep

\runsubsection{Department of Information Management, NCNU}
\descript{System Admin}
\null\hfill\location{June 2014 – Sep 2016 | Nantou}
\begin{tightemize}
    \item Converted all physical machines into virtual machines and seamlessly migrated them into the VMware vSphere environment.
    \item Migrated from an Apache Web Server to Nginx, setting up the Nginx server with HTTP/2 and SSL, resulting in a 60% improvement in page-loading time.
    \item Developed an automated deployment script for CMS platforms (such as WordPress or Joomla).
\end{tightemize}



\end{minipage}
%%%%%%%%%%%%%%%%%%%%%%%%%%%%%%%%%%%%%%
%
%     PAGE TWO
%
%%%%%%%%%%%%%%%%%%%%%%%%%%%%%%%%%%%%%%

\hfill
\begin{minipage}[t]{1\textwidth}


\section{Experience (Organization / Community)}
\runsubsection{MOLi - Makers' Open Lab for Innovation}
\descript{Event Coordinator}
\null\hfill\location{2015 – July 2019 | Nantou, Taiwan / Remote}
\vspace{\topsep}
\begin{tightemize}
    \item Coordinated over 10 MOLiDay sharing events, featuring various topics such as technology, education topics.
    \item Managed the organization's Facebook fan page, utilizing content creation and social media marketing strategies to engage with followers and promote events.
\end{tightemize}
\hspace{12pt}\keyword{KKTIX: \urlstyle{same}\href{https://moli.kktix.cc/}{https://moli.kktix.cc/}}
\sectionsep


\runsubsection{MOLi - Online study circle}
\descript{Host}
\null\hfill\location{Nov 2020 – May 2022 | Remote}
\begin{tightemize}
    \item \urlstyle{same}\href{https://lsa.moli.rocks/study-circle/docs/2020-11-07}{2020 - 《鳥哥的 Linux 私房菜:基礎學習篇》}
    \item \urlstyle{same}\href{https://lsa.moli.rocks/study-circle/docs/docker-1}{2021 - 《Docker in Action 2nd edition》}
    \item \urlstyle{same}\href{https://lsa.moli.rocks/study-circle/docs/k8s-1}{2022 - 《Kubernetes in Action》}
\end{tightemize}
\sectionsep

%%%%%%%%%%%%%%%%%%%%%%%%%%%%%%%%%%%%%%
%     CERTIFICATE
%%%%%%%%%%%%%%%%%%%%%%%%%%%%%%%%%%%%%%

\section{Certificate}

\runsubsection{\urlstyle{same}\href{https://www.credly.com/earner/earned/badge/935959f7-a8b7-4acf-91ef-0cff90eadb5f}{Certified Kubernetes Administrator(CKA)}}
\descript{by The Linux Foundation}
\null\hfill\location{Dec 2021 - Dec 3 2024}
\runsubsection{\urlstyle{same}\href{https://www.credly.com/badges/75c12266-f115-417e-8876-d43ee5e747cf/linked_in_profile}{EKS Subject Matter Expert}}
\descript{by AWS Support Engineering}
\null\hfill\location{Apr 2022 - present}

%%%%%%%%%%%%%%%%%%%%%%%%%%%%%%%%%%%%%%
%     PUBLICATIONS
%%%%%%%%%%%%%%%%%%%%%%%%%%%%%%%%%%%%%%
\section{Publications}

\runsubsection{2022 iThome 鐵人賽}
\descript{DevOps 組優選}
\null\hfill\location{Jan 2024 | Remote}
\begin{tightemize}
    \item 那些文件沒有告訴你的 AWS EKS:解析 Kubernetes 背後的奧秘(iThome鐵人賽系列書 - \urlstyle{same}\href{https://www.tenlong.com.tw/products/9786263337145}{https://www.tenlong.com.tw/products/9786263337145}
\end{tightemize}
\sectionsep

\runsubsection{MOLi - Online study circle}
\descript{Collaborative notes}
\null\hfill\location{Dec 2022 | Remote}
\begin{tightemize}
    \item Webpage - \urlstyle{same}\href{https://lsa.moli.rocks/study-circle/docs/}{https://lsa.moli.rocks/study-circle/docs/}
    \item HackMD - \urlstyle{same}\href{https://hackmd.io/@ncnu-opensource/linux-study-circle/}{https://hackmd.io/@ncnu-opensource/linux-study-circle/}
\end{tightemize}
\sectionsep

\runsubsection{NCNU OpenSource}
\descript{Collaborative notes}
\null\hfill\location{2014 - present | Remote}
\begin{tightemize}
    \item GitHub - \urlstyle{same}\href{https://github.com/NCNU-OpenSource}{https://github.com/NCNU-OpenSource}
    \item HackMD - \urlstyle{same}\href{https://hackmd.io/@ncnu-opensource/book}{https://hackmd.io/@ncnu-opensource/book}
\end{tightemize}

%%%%%%%%%%%%%%%%%%%%%%%%%%%%%%%%%%%%%%
%     Presentations
%%%%%%%%%%%%%%%%%%%%%%%%%%%%%%%%%%%%%%
\section{Presentations}

\runsubsection{\urlstyle{same}\href{https://itplus.ithome.com.tw/webinar-page/221}{深入淺出 EKS Pod Identities}}
\descript{iThome 鐵人講堂}

\runsubsection{\urlstyle{same}\href{https://hktw-resources.awscloud.com/aws-devax-amazon-eks}{Amazon EKS 與現代化應用管理者實踐}}
\descript{AWS 現代應用開發賦能系列}
\null\hfill\location{March 21 2024}

\runsubsection{\urlstyle{same}\href{https://docs.google.com/presentation/d/11VJsfF0M3VXfFgFrija6Me2fHk0GvOeCUCtLQF1ZzNo/edit?usp=sharing}{EKS 上的一些安全性常見問題}}
\descript{AWS User Group Taiwan - Taichung}
\null\hfill\location{March 16 2024}

\runsubsection{\urlstyle{same}\href{https://docs.google.com/presentation/d/1dOWEOtbpZa8TO-hHOND6NYjbYQ911VlNrYNjhYEpYQo/edit?usp=sharing}{EKS 上的一些常見問題}}
\descript{AWS User Group Taiwan - Taipei}
\null\hfill\location{March 14 2024}


%%%%%%%%%%%%%%%%%%%%%%%%%%%%%%%%%%%%%%
%     EDUCATION
%%%%%%%%%%%%%%%%%%%%%%%%%%%%%%%%%%%%%%
\section{Education}

\subsection{Master of Information Management}
\descript{National Taiwan University of Science and Technology}
\null\hfill\location{Sep 2016 - Aug 2018 | Nantou, Taiwan}
\thesis{On the Design and Implementation a Logging Management System that Satisfying Common Criteria Security Auditing Requirements with the Blockchain Technology}
\keyword{Keywords: Common Criteria, Security Auditing, Blockchain, Smart Contract, Hypereledger}

\vspace{\topsep}
\subsection{Bachelor of Information Management}
\descript{National Chi Nan University}
\null\hfill\location{Sep 2011 - Aug 2016 | Taipei, Taiwan}
\thesis{On the Design and Implementation Reaction Mechanisms of a Intrusion Detection and Prevention System in Linux Kernel-Level }
\keyword{Keywords: Virtual Machine, Virtual Machine Monitor, Intrusion Detection and Prevention System, Linux Kernel}
\end{minipage}

\end{document}  \documentclass[]{article}
